% Options for packages loaded elsewhere
\PassOptionsToPackage{unicode}{hyperref}
\PassOptionsToPackage{hyphens}{url}
\PassOptionsToPackage{dvipsnames,svgnames,x11names}{xcolor}
%
\documentclass[
]{article}
\usepackage{amsmath,amssymb}
\usepackage{lmodern}
\usepackage{iftex}
\ifPDFTeX
  \usepackage[T1]{fontenc}
  \usepackage[utf8]{inputenc}
  \usepackage{textcomp} % provide euro and other symbols
\else % if luatex or xetex
  \usepackage{unicode-math}
  \defaultfontfeatures{Scale=MatchLowercase}
  \defaultfontfeatures[\rmfamily]{Ligatures=TeX,Scale=1}
\fi
% Use upquote if available, for straight quotes in verbatim environments
\IfFileExists{upquote.sty}{\usepackage{upquote}}{}
\IfFileExists{microtype.sty}{% use microtype if available
  \usepackage[]{microtype}
  \UseMicrotypeSet[protrusion]{basicmath} % disable protrusion for tt fonts
}{}
\makeatletter
\@ifundefined{KOMAClassName}{% if non-KOMA class
  \IfFileExists{parskip.sty}{%
    \usepackage{parskip}
  }{% else
    \setlength{\parindent}{0pt}
    \setlength{\parskip}{6pt plus 2pt minus 1pt}}
}{% if KOMA class
  \KOMAoptions{parskip=half}}
\makeatother
\usepackage{xcolor}
\usepackage[margin=1in]{geometry}
\usepackage{color}
\usepackage{fancyvrb}
\newcommand{\VerbBar}{|}
\newcommand{\VERB}{\Verb[commandchars=\\\{\}]}
\DefineVerbatimEnvironment{Highlighting}{Verbatim}{commandchars=\\\{\}}
% Add ',fontsize=\small' for more characters per line
\usepackage{framed}
\definecolor{shadecolor}{RGB}{248,248,248}
\newenvironment{Shaded}{\begin{snugshade}}{\end{snugshade}}
\newcommand{\AlertTok}[1]{\textcolor[rgb]{0.94,0.16,0.16}{#1}}
\newcommand{\AnnotationTok}[1]{\textcolor[rgb]{0.56,0.35,0.01}{\textbf{\textit{#1}}}}
\newcommand{\AttributeTok}[1]{\textcolor[rgb]{0.77,0.63,0.00}{#1}}
\newcommand{\BaseNTok}[1]{\textcolor[rgb]{0.00,0.00,0.81}{#1}}
\newcommand{\BuiltInTok}[1]{#1}
\newcommand{\CharTok}[1]{\textcolor[rgb]{0.31,0.60,0.02}{#1}}
\newcommand{\CommentTok}[1]{\textcolor[rgb]{0.56,0.35,0.01}{\textit{#1}}}
\newcommand{\CommentVarTok}[1]{\textcolor[rgb]{0.56,0.35,0.01}{\textbf{\textit{#1}}}}
\newcommand{\ConstantTok}[1]{\textcolor[rgb]{0.00,0.00,0.00}{#1}}
\newcommand{\ControlFlowTok}[1]{\textcolor[rgb]{0.13,0.29,0.53}{\textbf{#1}}}
\newcommand{\DataTypeTok}[1]{\textcolor[rgb]{0.13,0.29,0.53}{#1}}
\newcommand{\DecValTok}[1]{\textcolor[rgb]{0.00,0.00,0.81}{#1}}
\newcommand{\DocumentationTok}[1]{\textcolor[rgb]{0.56,0.35,0.01}{\textbf{\textit{#1}}}}
\newcommand{\ErrorTok}[1]{\textcolor[rgb]{0.64,0.00,0.00}{\textbf{#1}}}
\newcommand{\ExtensionTok}[1]{#1}
\newcommand{\FloatTok}[1]{\textcolor[rgb]{0.00,0.00,0.81}{#1}}
\newcommand{\FunctionTok}[1]{\textcolor[rgb]{0.00,0.00,0.00}{#1}}
\newcommand{\ImportTok}[1]{#1}
\newcommand{\InformationTok}[1]{\textcolor[rgb]{0.56,0.35,0.01}{\textbf{\textit{#1}}}}
\newcommand{\KeywordTok}[1]{\textcolor[rgb]{0.13,0.29,0.53}{\textbf{#1}}}
\newcommand{\NormalTok}[1]{#1}
\newcommand{\OperatorTok}[1]{\textcolor[rgb]{0.81,0.36,0.00}{\textbf{#1}}}
\newcommand{\OtherTok}[1]{\textcolor[rgb]{0.56,0.35,0.01}{#1}}
\newcommand{\PreprocessorTok}[1]{\textcolor[rgb]{0.56,0.35,0.01}{\textit{#1}}}
\newcommand{\RegionMarkerTok}[1]{#1}
\newcommand{\SpecialCharTok}[1]{\textcolor[rgb]{0.00,0.00,0.00}{#1}}
\newcommand{\SpecialStringTok}[1]{\textcolor[rgb]{0.31,0.60,0.02}{#1}}
\newcommand{\StringTok}[1]{\textcolor[rgb]{0.31,0.60,0.02}{#1}}
\newcommand{\VariableTok}[1]{\textcolor[rgb]{0.00,0.00,0.00}{#1}}
\newcommand{\VerbatimStringTok}[1]{\textcolor[rgb]{0.31,0.60,0.02}{#1}}
\newcommand{\WarningTok}[1]{\textcolor[rgb]{0.56,0.35,0.01}{\textbf{\textit{#1}}}}
\usepackage{graphicx}
\makeatletter
\def\maxwidth{\ifdim\Gin@nat@width>\linewidth\linewidth\else\Gin@nat@width\fi}
\def\maxheight{\ifdim\Gin@nat@height>\textheight\textheight\else\Gin@nat@height\fi}
\makeatother
% Scale images if necessary, so that they will not overflow the page
% margins by default, and it is still possible to overwrite the defaults
% using explicit options in \includegraphics[width, height, ...]{}
\setkeys{Gin}{width=\maxwidth,height=\maxheight,keepaspectratio}
% Set default figure placement to htbp
\makeatletter
\def\fps@figure{htbp}
\makeatother
\setlength{\emergencystretch}{3em} % prevent overfull lines
\providecommand{\tightlist}{%
  \setlength{\itemsep}{0pt}\setlength{\parskip}{0pt}}
\setcounter{secnumdepth}{-\maxdimen} % remove section numbering
\ifLuaTeX
  \usepackage{selnolig}  % disable illegal ligatures
\fi
\IfFileExists{bookmark.sty}{\usepackage{bookmark}}{\usepackage{hyperref}}
\IfFileExists{xurl.sty}{\usepackage{xurl}}{} % add URL line breaks if available
\urlstyle{same} % disable monospaced font for URLs
\hypersetup{
  colorlinks=true,
  linkcolor={Maroon},
  filecolor={Maroon},
  citecolor={Blue},
  urlcolor={blue},
  pdfcreator={LaTeX via pandoc}}

\title{PES University, Bangalore

Established under the Karnataka Act No.~16 of 2013}
\usepackage{etoolbox}
\makeatletter
\providecommand{\subtitle}[1]{% add subtitle to \maketitle
  \apptocmd{\@title}{\par {\large #1 \par}}{}{}
}
\makeatother
\subtitle{\textbf{UE20CS312 - Data Analytics}

\textbf{Worksheet 1a - Part 1: Exploring Data with R}

Harshith Mohan Kumar -
\href{mailto:harshithmohankumar@pesu.pes.edu}{\nolinkurl{harshithmohankumar@pesu.pes.edu}}

Anushka Hebbar -
\href{mailto:anushkahebbar@pesu.pes.edu}{\nolinkurl{anushkahebbar@pesu.pes.edu}}}
\author{}
\date{\vspace{-2.5em}}

\begin{document}
\maketitle

\hypertarget{exploring-data-with-r}{%
\section{Exploring Data with R}\label{exploring-data-with-r}}

\hypertarget{prerequisites}{%
\subsection{Prerequisites}\label{prerequisites}}

This worksheet aims to develop your understanding of summary statistics
and basic visualizations through a pragmatic approach. To download the
data required for this worksheet, visit
\href{https://tinyurl.com/worksheet1a-part1}{this Github link}.

\textbf{Resources: }

\begin{itemize}
\tightlist
\item
  Check out \href{https://intro2r.com/}{this} beautifully comprehensive
  resource for everything you need to get started with R.
\item
  \href{https://r-graphics.org/}{This online book} provides guided
  explananations about visualizations in R using the \texttt{ggplot2}
  library.
\end{itemize}

\hypertarget{about-the-data-top-1000-instagrammers}{%
\subsection{1. About the Data: Top 1000
Instagrammers}\label{about-the-data-top-1000-instagrammers}}

To make this worksheet a bit interesting for you all, we have picked a
dataset from Kaggle which comprises of the details of the top 1000
influencers on Instagram. If you are on this list, send me an email ;P

This dataset has been taken from
\href{https://www.kaggle.com/datasets/syedjaferk/top-1000-instagrammers-world-cleaned?resource=download}{this
Kaggle dataset by Syed Jafer}.

\textbf{Data Dictionary}

\begin{verbatim}
Name: Name of the account
Rank: Overall rank in the world.
Category: Stream of the account (Music, Games, etc..)
Followers: Number of followers
Audience Country: country of the majority of audience.
Authentic Engagement: Engagement with the users.
Engagement Avg.: Average engagement of the users.
\end{verbatim}

\hypertarget{assignment-submission-format}{%
\subsection{2. Assignment Submission
Format}\label{assignment-submission-format}}

The following problems are to be completed using the R programming
language and should be submitted as a R markdown file (\texttt{.rmd}).
Since the dataset is public and many of you students will have the same
numerical answers, the grades are allocated on the analysis of the
problems and personalized answers within the conclusion section.

\pagebreak

The markdown file should follow this format:

\begin{Shaded}
\begin{Highlighting}[]
\CommentTok{{-}{-}{-}}
\AnnotationTok{title:}\CommentTok{ "UE20CS312 {-} Data Analytics {-} Worksheet 1a {-} Part 1 {-} Exploring data with R"}
\AnnotationTok{subtitle:}\CommentTok{ "PES University"}
\AnnotationTok{author:}\CommentTok{ }
\CommentTok{  {-} \textquotesingle{}INSERT\_NAME, Dept. of CSE {-} INSERT\_SRN\textquotesingle{}}
\AnnotationTok{output:}\CommentTok{ pdf\_document}
\AnnotationTok{urlcolor:}\CommentTok{ blue}
\AnnotationTok{editor\_options:}\CommentTok{ }
\CommentTok{  markdown: }
\CommentTok{    wrap: 72}
\CommentTok{{-}{-}{-}}

\FunctionTok{\#\# Solutions}

\FunctionTok{\#\#\# Problem 1}
\NormalTok{INSERT SOLUTION CODE IN MARKDOWN}
\NormalTok{INSERT SCREENSHOT OF R OUTPUT}
\NormalTok{INSERT ANALYSIS}

\FunctionTok{\#\#\# Problem 2}
\NormalTok{INSERT SOLUTION CODE IN MARKDOWN}
\NormalTok{INSERT SCREENSHOT OF R OUTPUT}
\NormalTok{INSERT ANALYSIS}
\NormalTok{(etc)}

\FunctionTok{\#\#\# Conclusion}
\NormalTok{INSERT SUMMARY}
\end{Highlighting}
\end{Shaded}

\hypertarget{loading-the-dataset}{%
\subsection{3. Loading the Dataset}\label{loading-the-dataset}}

\textbf{Step 1:} Ensure you are on the right working directory and the
CSV exists in this directory.

\begin{Shaded}
\begin{Highlighting}[]
\CommentTok{\# Get and print current working directory}
\FunctionTok{print}\NormalTok{(}\FunctionTok{getwd}\NormalTok{())}

\CommentTok{\# Set current working directory}
\FunctionTok{setwd}\NormalTok{(}\StringTok{"/UR/WORKING/DIRECTORY"}\NormalTok{)}

\CommentTok{\# Get and print current working directory}
\FunctionTok{print}\NormalTok{(}\FunctionTok{getwd}\NormalTok{())}
\end{Highlighting}
\end{Shaded}

\textbf{Step 2:} Read CSV File

\begin{Shaded}
\begin{Highlighting}[]
\CommentTok{\# Load CSV}
\NormalTok{data }\OtherTok{\textless{}{-}} \FunctionTok{read.csv}\NormalTok{(}\StringTok{"top\_1000\_instagrammers.csv"}\NormalTok{)}
\end{Highlighting}
\end{Shaded}

\hypertarget{preliminary-guided-exercises}{%
\subsection{4. Preliminary Guided
Exercises}\label{preliminary-guided-exercises}}

Make sure you have the R programming language installed on your system.
It is also recommended to make sure RStudio, the popular IDE for R, is
installed. RStudio provides a lot of useful functionality like R
markdown, a script editor and GitHub integration. Use RStudio Projects
as a great way of keeping each week's assignment work organized.

\pagebreak

\hypertarget{data-import}{%
\paragraph{Data Import}\label{data-import}}

To import data from CSV files into a DataFrame:

\begin{Shaded}
\begin{Highlighting}[]
\NormalTok{data }\OtherTok{\textless{}{-}} \FunctionTok{read.csv}\NormalTok{(}\StringTok{\textquotesingle{}top\_1000\_instagrammers.csv\textquotesingle{}}\NormalTok{, }\AttributeTok{header=}\ConstantTok{TRUE}\NormalTok{)}
\end{Highlighting}
\end{Shaded}

The \texttt{header\ =\ TRUE} argument specifies that the first row of
your data contains the variable names. If this is not the case you can
specify \texttt{header\ =\ FALSE} (this is the default value so you can
omit this argument entirely).

\hypertarget{compact-summary}{%
\paragraph{Compact Summary}\label{compact-summary}}

Use the \texttt{str()} function to return a compact and informative
summary of the DataFrame.

\begin{Shaded}
\begin{Highlighting}[]
\FunctionTok{str}\NormalTok{(data) }
\end{Highlighting}
\end{Shaded}

\begin{verbatim}
## 'data.frame':  1000 obs. of  8 variables:
##  $ X                   : int  0 1 2 3 4 5 6 7 8 9 ...
##  $ Name                : chr  "cristiano" "leomessi" "kendalljenner" "arianagrande" ...
##  $ Rank                : int  1 2 3 4 5 6 7 8 9 10 ...
##  $ Category            : chr  "Sports with a ball" "Sports with a ballFamily" "ModelingFashion" "Music" ...
##  $ Followers           : num  4.63e+08 3.47e+08 2.48e+08 3.21e+08 1.47e+08 ...
##  $ Audience.Country    : chr  "India" "Argentina" "United States" "United States" ...
##  $ Authentic.Engagement: num  5500000 3600000 3000000 2400000 4300000 1700000 2400000 1200000 1400000 13300000 ...
##  $ Engagement.Avg.     : num  6600000 4800000 4900000 3400000 5800000 2500000 3200000 1900000 1900000 13300000 ...
\end{verbatim}

Here we see that flowers is a `data.frame' object which contains 1000
rows and 8 variables (columns). Each of the variables are listed along
with their data class and the first 10 values.

\hypertarget{summary-statistics}{%
\paragraph{Summary Statistics}\label{summary-statistics}}

To access the data in any of the variables (columns) in our data frame
we can use the \$ notation. Indexing in R starts at 1, which means the
first element is at index 1. Access the first 10 values of the
\texttt{Name} column:

\begin{Shaded}
\begin{Highlighting}[]
\NormalTok{data}\SpecialCharTok{$}\NormalTok{Name[}\DecValTok{1}\SpecialCharTok{:}\DecValTok{10}\NormalTok{]}
\end{Highlighting}
\end{Shaded}

\begin{verbatim}
##  [1] "cristiano"     "leomessi"      "kendalljenner" "arianagrande" 
##  [5] "zendaya"       "kimkardashian" "taylorswift"   "kyliejenner"  
##  [9] "selenagomez"   "thv"
\end{verbatim}

We can assign a column to another variable and calculate a mean of a
numeric variable or get a summary of a variable using the summary()
function.

\begin{Shaded}
\begin{Highlighting}[]
\NormalTok{names }\OtherTok{\textless{}{-}}\NormalTok{ data}\SpecialCharTok{$}\NormalTok{Name}
\FunctionTok{summary}\NormalTok{(names)}
\end{Highlighting}
\end{Shaded}

\begin{verbatim}
##    Length     Class      Mode 
##      1000 character character
\end{verbatim}

\begin{Shaded}
\begin{Highlighting}[]
\NormalTok{auth\_eng }\OtherTok{\textless{}{-}}\NormalTok{ data}\SpecialCharTok{$}\NormalTok{Authentic.Engagement}
\FunctionTok{mean}\NormalTok{(auth\_eng)}
\end{Highlighting}
\end{Shaded}

\begin{verbatim}
## [1] 566199.2
\end{verbatim}

\begin{Shaded}
\begin{Highlighting}[]
\FunctionTok{summary}\NormalTok{(auth\_eng)}
\end{Highlighting}
\end{Shaded}

\begin{verbatim}
##     Min.  1st Qu.   Median     Mean  3rd Qu.     Max. 
##        0   169000   316050   566199   604275 13300000
\end{verbatim}

Notice how the behavior of the \texttt{summary} function changes with
different types of variables. Let's now try to explore how we can
visualize our data.

\pagebreak

\hypertarget{scatterplots}{%
\paragraph{Scatterplots}\label{scatterplots}}

The most common high level function used to produce plots in R is the
\texttt{plot} function.

\begin{Shaded}
\begin{Highlighting}[]
\FunctionTok{plot}\NormalTok{(data}\SpecialCharTok{$}\NormalTok{Engagement.Avg., }\AttributeTok{type=}\StringTok{"p"}\NormalTok{)}
\FunctionTok{plot}\NormalTok{(data}\SpecialCharTok{$}\NormalTok{Authentic.Engagement, }\AttributeTok{type=}\StringTok{"l"}\NormalTok{)}
\end{Highlighting}
\end{Shaded}

\includegraphics[width=0.5\linewidth]{Worksheet_1a_Part1_files/figure-latex/unnamed-chunk-5-1}
\includegraphics[width=0.5\linewidth]{Worksheet_1a_Part1_files/figure-latex/unnamed-chunk-5-2}

\hypertarget{using-the-ggplot2-library}{%
\paragraph{\texorpdfstring{Using the \texttt{ggplot2}
Library}{Using the ggplot2 Library}}\label{using-the-ggplot2-library}}

Create a bar graph for the distribution of the categorical variable
\texttt{Audience.Country}.

\begin{Shaded}
\begin{Highlighting}[]
\CommentTok{\# Import the library for visualization}
\FunctionTok{library}\NormalTok{(ggplot2) }
\CommentTok{\# Create a bar graph}
\FunctionTok{ggplot}\NormalTok{(data, }\FunctionTok{aes}\NormalTok{(}\AttributeTok{x=}\NormalTok{Audience.Country)) }\SpecialCharTok{+} \FunctionTok{geom\_bar}\NormalTok{() }\SpecialCharTok{+} \FunctionTok{coord\_flip}\NormalTok{()}
\CommentTok{\# Number of followers of the top 10 most followed instagrammers}
\FunctionTok{ggplot}\NormalTok{(data[}\DecValTok{1}\SpecialCharTok{:}\DecValTok{10}\NormalTok{,], }\FunctionTok{aes}\NormalTok{(}\AttributeTok{x=}\NormalTok{Name, }\AttributeTok{y=}\NormalTok{Followers)) }\SpecialCharTok{+} \FunctionTok{geom\_col}\NormalTok{() }\SpecialCharTok{+} \FunctionTok{coord\_flip}\NormalTok{()}
\end{Highlighting}
\end{Shaded}

\includegraphics[width=0.5\linewidth]{Worksheet_1a_Part1_files/figure-latex/unnamed-chunk-6-1}
\includegraphics[width=0.5\linewidth]{Worksheet_1a_Part1_files/figure-latex/unnamed-chunk-6-2}

\hypertarget{points}{%
\subsection{5. Points}\label{points}}

The problems for this part of the worksheet are for a total of 8 points,
with a non-uniform weightage.

\begin{itemize}
\tightlist
\item
  \emph{Problem 1} : 1 point
\item
  \emph{Problem 2} : 2 points
\item
  \emph{Problem 3} : 1 points
\item
  \emph{Problem 4} : 3 point
\item
  \emph{Conclusion} : 1 point
\end{itemize}

\hypertarget{problems}{%
\subsection{6. Problems}\label{problems}}

\hypertarget{problem-1-1-point}{%
\subsubsection{Problem 1 (1 point)}\label{problem-1-1-point}}

Get the summary statistics (mean, median, mode, min, max, 1st quartile,
3rd quartile and standard deviation) for the dataset. Calculate these
only for the numerical columns {[}Audience Country, Authentic Engagement
and Engagement Average{]}. What can you determine from the summary
statistics? How does your Instagram stats hold up with the top 1000 :P ?

\hypertarget{problem-2-2-points}{%
\subsubsection{Problem 2 (2 points)}\label{problem-2-2-points}}

What are the top 3 audience countries that follow most of the top 1000
instagrammers? \emph{Hint:} Go back to bar graph created earlier. Use R
to calculate the percentage of the top 1000 instagrammers that have the
top 1 audience country.

\hypertarget{problem-3-1-point}{%
\subsubsection{Problem 3 (1 point)}\label{problem-3-1-point}}

Create a horizontal box plot using the column
\texttt{Authentic.Engagement}. What inferences can you make from this
box and whisker plot?

\hypertarget{problem-4-3-points}{%
\subsubsection{Problem 4 (3 points)}\label{problem-4-3-points}}

Create a histogram where the x-axis contains the Audience Country and
y-axis contains the total follower count for accounts with that Audience
Country. Which country is associated with the most amount of followers?
\emph{Hint:} Recall the concept of \texttt{groupby()} in Pandas. Try
using the \texttt{aggregate()} function in R to achieve the same goal.
What is the total for India and what rank does it fall compared to other
countries?

\hypertarget{conclusion-1-point}{%
\subsection{Conclusion (1 point)}\label{conclusion-1-point}}

In a few short sentences, describe your Instagram profile (category,
followers, estimated engagement). Compare your profile to the analysis
done of the top 1000 profiles. If you were tasked to becoming an
influencer, what would be the best way for you to increase your
followers and user engagement?

\end{document}

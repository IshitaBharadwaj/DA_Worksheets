% Options for packages loaded elsewhere
\PassOptionsToPackage{unicode}{hyperref}
\PassOptionsToPackage{hyphens}{url}
\PassOptionsToPackage{dvipsnames,svgnames,x11names}{xcolor}
%
\documentclass[
]{article}
\usepackage{amsmath,amssymb}
\usepackage{lmodern}
\usepackage{iftex}
\ifPDFTeX
  \usepackage[T1]{fontenc}
  \usepackage[utf8]{inputenc}
  \usepackage{textcomp} % provide euro and other symbols
\else % if luatex or xetex
  \usepackage{unicode-math}
  \defaultfontfeatures{Scale=MatchLowercase}
  \defaultfontfeatures[\rmfamily]{Ligatures=TeX,Scale=1}
\fi
% Use upquote if available, for straight quotes in verbatim environments
\IfFileExists{upquote.sty}{\usepackage{upquote}}{}
\IfFileExists{microtype.sty}{% use microtype if available
  \usepackage[]{microtype}
  \UseMicrotypeSet[protrusion]{basicmath} % disable protrusion for tt fonts
}{}
\makeatletter
\@ifundefined{KOMAClassName}{% if non-KOMA class
  \IfFileExists{parskip.sty}{%
    \usepackage{parskip}
  }{% else
    \setlength{\parindent}{0pt}
    \setlength{\parskip}{6pt plus 2pt minus 1pt}}
}{% if KOMA class
  \KOMAoptions{parskip=half}}
\makeatother
\usepackage{xcolor}
\usepackage[margin=1in]{geometry}
\usepackage{color}
\usepackage{fancyvrb}
\newcommand{\VerbBar}{|}
\newcommand{\VERB}{\Verb[commandchars=\\\{\}]}
\DefineVerbatimEnvironment{Highlighting}{Verbatim}{commandchars=\\\{\}}
% Add ',fontsize=\small' for more characters per line
\usepackage{framed}
\definecolor{shadecolor}{RGB}{248,248,248}
\newenvironment{Shaded}{\begin{snugshade}}{\end{snugshade}}
\newcommand{\AlertTok}[1]{\textcolor[rgb]{0.94,0.16,0.16}{#1}}
\newcommand{\AnnotationTok}[1]{\textcolor[rgb]{0.56,0.35,0.01}{\textbf{\textit{#1}}}}
\newcommand{\AttributeTok}[1]{\textcolor[rgb]{0.77,0.63,0.00}{#1}}
\newcommand{\BaseNTok}[1]{\textcolor[rgb]{0.00,0.00,0.81}{#1}}
\newcommand{\BuiltInTok}[1]{#1}
\newcommand{\CharTok}[1]{\textcolor[rgb]{0.31,0.60,0.02}{#1}}
\newcommand{\CommentTok}[1]{\textcolor[rgb]{0.56,0.35,0.01}{\textit{#1}}}
\newcommand{\CommentVarTok}[1]{\textcolor[rgb]{0.56,0.35,0.01}{\textbf{\textit{#1}}}}
\newcommand{\ConstantTok}[1]{\textcolor[rgb]{0.00,0.00,0.00}{#1}}
\newcommand{\ControlFlowTok}[1]{\textcolor[rgb]{0.13,0.29,0.53}{\textbf{#1}}}
\newcommand{\DataTypeTok}[1]{\textcolor[rgb]{0.13,0.29,0.53}{#1}}
\newcommand{\DecValTok}[1]{\textcolor[rgb]{0.00,0.00,0.81}{#1}}
\newcommand{\DocumentationTok}[1]{\textcolor[rgb]{0.56,0.35,0.01}{\textbf{\textit{#1}}}}
\newcommand{\ErrorTok}[1]{\textcolor[rgb]{0.64,0.00,0.00}{\textbf{#1}}}
\newcommand{\ExtensionTok}[1]{#1}
\newcommand{\FloatTok}[1]{\textcolor[rgb]{0.00,0.00,0.81}{#1}}
\newcommand{\FunctionTok}[1]{\textcolor[rgb]{0.00,0.00,0.00}{#1}}
\newcommand{\ImportTok}[1]{#1}
\newcommand{\InformationTok}[1]{\textcolor[rgb]{0.56,0.35,0.01}{\textbf{\textit{#1}}}}
\newcommand{\KeywordTok}[1]{\textcolor[rgb]{0.13,0.29,0.53}{\textbf{#1}}}
\newcommand{\NormalTok}[1]{#1}
\newcommand{\OperatorTok}[1]{\textcolor[rgb]{0.81,0.36,0.00}{\textbf{#1}}}
\newcommand{\OtherTok}[1]{\textcolor[rgb]{0.56,0.35,0.01}{#1}}
\newcommand{\PreprocessorTok}[1]{\textcolor[rgb]{0.56,0.35,0.01}{\textit{#1}}}
\newcommand{\RegionMarkerTok}[1]{#1}
\newcommand{\SpecialCharTok}[1]{\textcolor[rgb]{0.00,0.00,0.00}{#1}}
\newcommand{\SpecialStringTok}[1]{\textcolor[rgb]{0.31,0.60,0.02}{#1}}
\newcommand{\StringTok}[1]{\textcolor[rgb]{0.31,0.60,0.02}{#1}}
\newcommand{\VariableTok}[1]{\textcolor[rgb]{0.00,0.00,0.00}{#1}}
\newcommand{\VerbatimStringTok}[1]{\textcolor[rgb]{0.31,0.60,0.02}{#1}}
\newcommand{\WarningTok}[1]{\textcolor[rgb]{0.56,0.35,0.01}{\textbf{\textit{#1}}}}
\usepackage{graphicx}
\makeatletter
\def\maxwidth{\ifdim\Gin@nat@width>\linewidth\linewidth\else\Gin@nat@width\fi}
\def\maxheight{\ifdim\Gin@nat@height>\textheight\textheight\else\Gin@nat@height\fi}
\makeatother
% Scale images if necessary, so that they will not overflow the page
% margins by default, and it is still possible to overwrite the defaults
% using explicit options in \includegraphics[width, height, ...]{}
\setkeys{Gin}{width=\maxwidth,height=\maxheight,keepaspectratio}
% Set default figure placement to htbp
\makeatletter
\def\fps@figure{htbp}
\makeatother
\setlength{\emergencystretch}{3em} % prevent overfull lines
\providecommand{\tightlist}{%
  \setlength{\itemsep}{0pt}\setlength{\parskip}{0pt}}
\setcounter{secnumdepth}{-\maxdimen} % remove section numbering
\ifLuaTeX
  \usepackage{selnolig}  % disable illegal ligatures
\fi
\IfFileExists{bookmark.sty}{\usepackage{bookmark}}{\usepackage{hyperref}}
\IfFileExists{xurl.sty}{\usepackage{xurl}}{} % add URL line breaks if available
\urlstyle{same} % disable monospaced font for URLs
\hypersetup{
  pdftitle={PES University, Bangalore},
  pdfauthor={UE20CS312 - Data Analytics - Worksheet 1b - Correlation Analysis; Designed by Vibha Masti, Dept. of CSE - vibha@pesu.pes.edu},
  colorlinks=true,
  linkcolor={Maroon},
  filecolor={Maroon},
  citecolor={Blue},
  urlcolor={blue},
  pdfcreator={LaTeX via pandoc}}

\title{PES University, Bangalore}
\usepackage{etoolbox}
\makeatletter
\providecommand{\subtitle}[1]{% add subtitle to \maketitle
  \apptocmd{\@title}{\par {\large #1 \par}}{}{}
}
\makeatother
\subtitle{Established under Karnataka Act No.~16 of 2013}
\author{UE20CS312 - Data Analytics - Worksheet 1b - Correlation
Analysis \and Designed by Vibha Masti, Dept. of CSE -
\href{mailto:vibha@pesu.pes.edu}{\nolinkurl{vibha@pesu.pes.edu}}}
\date{}

\begin{document}
\maketitle

\hypertarget{correlation}{%
\subsection{Correlation}\label{correlation}}

Correlation is a measure of the strength and direction of relationship
that exists between two random variables and is measured using
correlation coefficient. Correlation can assist data scientists to
choose the variables for model building that is used for solving an
analytics problem.

There are different types of correlation coefficients, based on the
nature of the data being compared:

\begin{itemize}
\tightlist
\item
  Between two continuous (interval, ratio) random variables -
  \emph{Pearson's Product Moment Correlation Coefficient}
\item
  Between two ordinal random variables - \emph{Spearman-Rank Correlation
  Coefficient}
\item
  Between a continuous RV and a dichotomous RV - \emph{Point Bi-Serial
  Correlation Coefficient}
\item
  Between two binary random variables - \emph{Phi Coefficient}
\end{itemize}

\hypertarget{road-accidents}{%
\subsubsection{Road Accidents}\label{road-accidents}}

India is the world's second-most populous country with a population of
around 1.2 billion people (as of July 2022). Roads are a very important
mode of transport in India, spanning over 6.2 million kilometers of
length, making it the country with the second-largest road network,
after the United States of America. (Source:
\href{https://en.wikipedia.org/wiki/Roads_in_India}{Wikipedia}). With
India trying to modernize its road infrastructure, there is still the
problem of frequent road accidents.

Road accidents in India is a major cause of death and injury. The NCRB
(National Crime Records Bureau) of India collects detailed data on
traffic accidents and collisions annually. Please download the dataset
from the
\href{https://github.com/Data-Analytics-UE20CS312/Unit-1-Worksheets/blob/main/1b\%20-\%20Correlation\%20Analysis/road_accidents_india_2016.csv}{GitHub
repository} that contains road accident data in India from 2016. The
data was obtained from
\href{https://www.kaggle.com/datasets/greeshmagirish/road-accidents-in-india-20142017}{this
kaggle dataset}.

\textbf{Data Dictionary}

\begin{verbatim}
S. No.: Serial number
State/ UT: name of state/union terrirory in India
Fine/Clear - Total Accidents: total accidents per state/UT in Fine/Clear weather conditions
Fine/Clear - Persons Killed: total fatalities per state/UT in Fine/Clear weather conditions
Fine/Clear - Persons Injured: total injured people per state/UT in Fine/Clear weather conditions
Mist/ Foggy - Total Accidents: total accidents per state/UT in Mist/Foggy weather conditions
Mist/ Foggy - Persons Killed: total fatalities per state/UT in Mist/Foggy weather conditions
Mist/ Foggy - Persons Injured: total injured people per state/UT in Mist/Foggy weather conditions
Cloudy - Total Accidents: total accidents per state/UT in Cloudy weather conditions
Cloudy - Persons Killed: total fatalities per state/UT in Cloudy weather conditions
Cloudy - Persons Injured: total injured people per state/UT in Cloudy weather conditions
Rainy - Total Accidents: total accidents per state/UT in Rainy weather conditions
Rainy - Persons Killed: total fatalities per state/UT in Rainy weather conditions
Rainy - Persons Injured: total injured people per state/UT in Rainy weather conditions
Snowfall - Total Accidents: total accidents per state/UT in Snowfall weather conditions
Snowfall - Persons Killed: total fatalities per state/UT in Snowfall weather conditions
Snowfall - Persons Injured: total injured people per state/UT in Snowfall weather conditions
Hail/Sleet - Total Accidents: total accidents per state/UT in Hail/Sleet weather conditions
Hail/Sleet - Persons Killed: total fatalities per state/UT in Hail/Sleet weather conditions
Hail/Sleet - Persons Injured: total injured people per state/UT in Hail/Sleet weather conditions
Dust Storm - Total Accidents: total accidents per state/UT in Dust Storm weather conditions
Dust Storm - Persons Killed: total fatalities per state/UT in Dust Storm weather conditions
Dust Storm - Persons Injured: total injured people per state/UT in Dust Storm weather conditions
Others - Total Accidents: total accidents per state/UT in Other weather conditions
Others - Persons Killed: total fatalities per state/UT in Other weather conditions
Others - Persons Injured: total injured people per state/UT in Other weather conditions
\end{verbatim}

\hypertarget{points}{%
\subsubsection{Points}\label{points}}

The problems in this worksheet are for a total of 10 points with each
problem having a different weightage.

\begin{itemize}
\tightlist
\item
  \emph{Problem 1}: 2 points
\item
  \emph{Problem 2}: 2 points
\item
  \emph{Problem 3}: 3 points
\item
  \emph{Problem 4}: 1.5 points
\item
  \emph{Problem 5}: 1.5 points
\end{itemize}

\hypertarget{problem-1-2-points}{%
\subsubsection{Problem 1 (2 points)}\label{problem-1-2-points}}

Find the total number of accidents in each state for the year 2016 and
display your results. Make sure to display all rows while printing the
dataframe. Print only the necessary columns. (Hint: use the grep command
to help filter out column names).

\begin{Shaded}
\begin{Highlighting}[]
\FunctionTok{library}\NormalTok{(ggpubr)}
\FunctionTok{library}\NormalTok{(dplyr)}
\NormalTok{df }\OtherTok{\textless{}{-}} \FunctionTok{read.csv}\NormalTok{(}\StringTok{\textquotesingle{}road\_accidents\_india\_2016.csv\textquotesingle{}}\NormalTok{, }\AttributeTok{row.names=}\DecValTok{1}\NormalTok{)}
\NormalTok{acc\_cols}\OtherTok{\textless{}{-}}\FunctionTok{grep}\NormalTok{(}\StringTok{"Total accidents"}\NormalTok{, }\FunctionTok{colnames}\NormalTok{(df), }\AttributeTok{ignore.case =} \ConstantTok{TRUE}\NormalTok{, }\AttributeTok{value=}\ConstantTok{TRUE}\NormalTok{)}
\FunctionTok{print}\NormalTok{(acc\_cols)}
\CommentTok{\#totalAccidents\textless{}{-}data.frame(state..ut=df$State..UT, total\_acc=rowSums(df[ ,c(acc\_cols)], na.rm=TRUE))}
\end{Highlighting}
\end{Shaded}

\hypertarget{problem-2-2-points}{%
\subsubsection{Problem 2 (2 points)}\label{problem-2-2-points}}

Find the
(\(\text{fatality rate} = \dfrac{\text{total number of deaths}}{\text{total number of accidents}}\))
in each state. Find out if there is a significant linear correlation at
a significance of \(\alpha=0.05\) between the \emph{fatality rate} of a
state and the \emph{mist/foggy rate} (fraction of total accidents that
happen in mist/foggy conditions).

Correlation between two continuous RVs: Pearson's correlation
coefficient. Pearson's correlation coefficient between two RVs \(x\) and
\(y\) is given by:

\[
  \rho = \frac{\text{Covariance}(x, y)}{\sigma_x \sigma_y}
\]

where \(\sigma\) is the standard deviation of a variable.

Plot the fatality rate against the mist/foggy rate. (Hint: use the
\texttt{ggscatter} library to plot a scatterplot with the confidence
interval of the correlation coefficient).

Plot the fatality rate and mist/foggy rate (see
\href{https://r-graph-gallery.com/13-scatter-plot.html}{this} and
\href{https://r-graph-gallery.com/6-graph-parameters-reminder.html}{this}
for R plot customization).

\hypertarget{problem-3-3-points}{%
\subsubsection{Problem 3 (3 points)}\label{problem-3-3-points}}

Rank the states based on total accidents and total fatalities (give a
rank of 1 to the state that has the highest value of a property). You
are free to use any tie-breaking method for assigning ranks.

Find the Spearman-Rank correlation coefficient between the two rank
columns and determine if there is any statistical significance at a
significance level of \(\alpha=0.05\). Also test the hypothesis that the
correlation coefficient is at least \(0.2\).

The t statistic is given by

\[
t = \dfrac{r_s - \rho_s}{\sqrt{\dfrac{1-r_s^2}{n-2}}}
\]

Where \(r_s\) is the calculated Spearman-Rank correlation coefficient
and \(\rho_s\) is the value of the population correlation coefficient
being tested against.

\hypertarget{problem-4-1.5-points}{%
\subsubsection{Problem 4 (1.5 points)}\label{problem-4-1.5-points}}

Convert the column \texttt{Hail.Sleet...Total.Accidents} to a binary
column as follows. If a hail/sleet accident has occurred in a state,
give that state a value of 1. Otherwise, give it a value of 0. Once
converted, find out if there is a significant correlation between the
\texttt{hail\_accident\_occcur} binary column created and the number of
rainy total accidents for every state.

Calculate the point bi-serial correlation coefficient between the two
columns. (Hint: it is equivalent to calculating the Pearson correlation
between a continuous and a dichotomous variable. You could also use the
\texttt{ltm} package's \texttt{biserial.cor} function).

\hypertarget{problem-5-1.5-points}{%
\subsubsection{Problem 5 (1.5 points)}\label{problem-5-1.5-points}}

Similar to in Problem 4, create a binary column to represent whether a
dust storm accident has occurred in a state (1 = occurred, 0 = not
occurred). Convert the two columns into a contingency table.

Calculate the phi coefficient of the two tables. (Hint: use the
\texttt{psych} package).

\end{document}

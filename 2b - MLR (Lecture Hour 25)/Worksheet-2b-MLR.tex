% Options for packages loaded elsewhere
\PassOptionsToPackage{unicode}{hyperref}
\PassOptionsToPackage{hyphens}{url}
\PassOptionsToPackage{dvipsnames,svgnames,x11names}{xcolor}
%
\documentclass[
]{article}
\usepackage{amsmath,amssymb}
\usepackage{lmodern}
\usepackage{iftex}
\ifPDFTeX
  \usepackage[T1]{fontenc}
  \usepackage[utf8]{inputenc}
  \usepackage{textcomp} % provide euro and other symbols
\else % if luatex or xetex
  \usepackage{unicode-math}
  \defaultfontfeatures{Scale=MatchLowercase}
  \defaultfontfeatures[\rmfamily]{Ligatures=TeX,Scale=1}
\fi
% Use upquote if available, for straight quotes in verbatim environments
\IfFileExists{upquote.sty}{\usepackage{upquote}}{}
\IfFileExists{microtype.sty}{% use microtype if available
  \usepackage[]{microtype}
  \UseMicrotypeSet[protrusion]{basicmath} % disable protrusion for tt fonts
}{}
\makeatletter
\@ifundefined{KOMAClassName}{% if non-KOMA class
  \IfFileExists{parskip.sty}{%
    \usepackage{parskip}
  }{% else
    \setlength{\parindent}{0pt}
    \setlength{\parskip}{6pt plus 2pt minus 1pt}}
}{% if KOMA class
  \KOMAoptions{parskip=half}}
\makeatother
\usepackage{xcolor}
\usepackage[margin=1in]{geometry}
\usepackage{color}
\usepackage{fancyvrb}
\newcommand{\VerbBar}{|}
\newcommand{\VERB}{\Verb[commandchars=\\\{\}]}
\DefineVerbatimEnvironment{Highlighting}{Verbatim}{commandchars=\\\{\}}
% Add ',fontsize=\small' for more characters per line
\usepackage{framed}
\definecolor{shadecolor}{RGB}{248,248,248}
\newenvironment{Shaded}{\begin{snugshade}}{\end{snugshade}}
\newcommand{\AlertTok}[1]{\textcolor[rgb]{0.94,0.16,0.16}{#1}}
\newcommand{\AnnotationTok}[1]{\textcolor[rgb]{0.56,0.35,0.01}{\textbf{\textit{#1}}}}
\newcommand{\AttributeTok}[1]{\textcolor[rgb]{0.77,0.63,0.00}{#1}}
\newcommand{\BaseNTok}[1]{\textcolor[rgb]{0.00,0.00,0.81}{#1}}
\newcommand{\BuiltInTok}[1]{#1}
\newcommand{\CharTok}[1]{\textcolor[rgb]{0.31,0.60,0.02}{#1}}
\newcommand{\CommentTok}[1]{\textcolor[rgb]{0.56,0.35,0.01}{\textit{#1}}}
\newcommand{\CommentVarTok}[1]{\textcolor[rgb]{0.56,0.35,0.01}{\textbf{\textit{#1}}}}
\newcommand{\ConstantTok}[1]{\textcolor[rgb]{0.00,0.00,0.00}{#1}}
\newcommand{\ControlFlowTok}[1]{\textcolor[rgb]{0.13,0.29,0.53}{\textbf{#1}}}
\newcommand{\DataTypeTok}[1]{\textcolor[rgb]{0.13,0.29,0.53}{#1}}
\newcommand{\DecValTok}[1]{\textcolor[rgb]{0.00,0.00,0.81}{#1}}
\newcommand{\DocumentationTok}[1]{\textcolor[rgb]{0.56,0.35,0.01}{\textbf{\textit{#1}}}}
\newcommand{\ErrorTok}[1]{\textcolor[rgb]{0.64,0.00,0.00}{\textbf{#1}}}
\newcommand{\ExtensionTok}[1]{#1}
\newcommand{\FloatTok}[1]{\textcolor[rgb]{0.00,0.00,0.81}{#1}}
\newcommand{\FunctionTok}[1]{\textcolor[rgb]{0.00,0.00,0.00}{#1}}
\newcommand{\ImportTok}[1]{#1}
\newcommand{\InformationTok}[1]{\textcolor[rgb]{0.56,0.35,0.01}{\textbf{\textit{#1}}}}
\newcommand{\KeywordTok}[1]{\textcolor[rgb]{0.13,0.29,0.53}{\textbf{#1}}}
\newcommand{\NormalTok}[1]{#1}
\newcommand{\OperatorTok}[1]{\textcolor[rgb]{0.81,0.36,0.00}{\textbf{#1}}}
\newcommand{\OtherTok}[1]{\textcolor[rgb]{0.56,0.35,0.01}{#1}}
\newcommand{\PreprocessorTok}[1]{\textcolor[rgb]{0.56,0.35,0.01}{\textit{#1}}}
\newcommand{\RegionMarkerTok}[1]{#1}
\newcommand{\SpecialCharTok}[1]{\textcolor[rgb]{0.00,0.00,0.00}{#1}}
\newcommand{\SpecialStringTok}[1]{\textcolor[rgb]{0.31,0.60,0.02}{#1}}
\newcommand{\StringTok}[1]{\textcolor[rgb]{0.31,0.60,0.02}{#1}}
\newcommand{\VariableTok}[1]{\textcolor[rgb]{0.00,0.00,0.00}{#1}}
\newcommand{\VerbatimStringTok}[1]{\textcolor[rgb]{0.31,0.60,0.02}{#1}}
\newcommand{\WarningTok}[1]{\textcolor[rgb]{0.56,0.35,0.01}{\textbf{\textit{#1}}}}
\usepackage{graphicx}
\makeatletter
\def\maxwidth{\ifdim\Gin@nat@width>\linewidth\linewidth\else\Gin@nat@width\fi}
\def\maxheight{\ifdim\Gin@nat@height>\textheight\textheight\else\Gin@nat@height\fi}
\makeatother
% Scale images if necessary, so that they will not overflow the page
% margins by default, and it is still possible to overwrite the defaults
% using explicit options in \includegraphics[width, height, ...]{}
\setkeys{Gin}{width=\maxwidth,height=\maxheight,keepaspectratio}
% Set default figure placement to htbp
\makeatletter
\def\fps@figure{htbp}
\makeatother
\setlength{\emergencystretch}{3em} % prevent overfull lines
\providecommand{\tightlist}{%
  \setlength{\itemsep}{0pt}\setlength{\parskip}{0pt}}
\setcounter{secnumdepth}{-\maxdimen} % remove section numbering
\ifLuaTeX
  \usepackage{selnolig}  % disable illegal ligatures
\fi
\IfFileExists{bookmark.sty}{\usepackage{bookmark}}{\usepackage{hyperref}}
\IfFileExists{xurl.sty}{\usepackage{xurl}}{} % add URL line breaks if available
\urlstyle{same} % disable monospaced font for URLs
\hypersetup{
  pdftitle={PES University, Bangalore},
  colorlinks=true,
  linkcolor={Maroon},
  filecolor={Maroon},
  citecolor={Blue},
  urlcolor={blue},
  pdfcreator={LaTeX via pandoc}}

\title{PES University, Bangalore}
\usepackage{etoolbox}
\makeatletter
\providecommand{\subtitle}[1]{% add subtitle to \maketitle
  \apptocmd{\@title}{\par {\large #1 \par}}{}{}
}
\makeatother
\subtitle{\textbf{UE20CS312 - Data Analytics}

\textbf{Worksheet 2b : Multiple Linear Regression}

Course Anchor : Dr.~Gowri Srinivasa

Prepared by : Nishanth M S -
\href{mailto:nishanthmsathish.23@gmail.com}{\nolinkurl{nishanthmsathish.23@gmail.com}}}
\author{}
\date{\vspace{-2.5em}}

\begin{document}
\maketitle

\hypertarget{multiple-linear-regression}{%
\subsection{Multiple Linear
Regression}\label{multiple-linear-regression}}

Multiple Linear Regression (MLR) is a statistical technique that uses
several explanatory variables to predict the outcome of response
variable.The goal of MLR is to model \textbf{a linear relationship}
between explanatory(independent) variables and response(dependent)
variables.

\hypertarget{data-dictionary}{%
\subsection{Data Dictionary}\label{data-dictionary}}

The data required for this worksheet can be downloaded
\href{https://github.com/Data-Analytics-UE20CS312/Unit-2-Worksheets/tree/main/2b\%20-\%20Multi\%20Linear\%20Regression}{from
this GitHub Link}. The data was obtained from
\href{https://www.kaggle.com/datasets/bricevergnou/spotify-recommendation}{this}
dataset from Kaggle. The dataset contains features of songs on Spotify
collected using Spotify API.The features are as follows :

-\textbf{acousticness} : A confidence measure from 0.0 to 1.0 of whether
the track is acoustic. 1.0 represents high confidence the track is
acoustic.

-\textbf{danceability} : Danceability describes how suitable a track is
for dancing based on a combination of musical elements including tempo,
rhythm stability, beat strength, and overall regularity. A value of 0.0
is least danceable and 1.0 is most danceable.

-\textbf{duration\_ms} : The duration of track in milliseconds.

-\textbf{energy} : Energy is a measure from 0.0 to 1.0 and represents a
perceptual measure of intensity and activity.Perceptual features
contributing to this attribute include dynamic range, perceived
loudness, timbre, onset rate, and general entropy.

-\textbf{instrumentalness} : Predicts whether a track contains no
vocals.The closer the instrumentalness value is to 1.0, the greater
likelihood the track contains no vocal content. Values above 0.5 are
intended to represent instrumental tracks, but confidence is higher as
the value approaches 1.0.

-\textbf{key} : The key the track is in. Integers map to pitches using
standard Pitch Class notation

-\textbf{liveness} : Detects the presence of an audience in the
recording. Higher liveness values represent an increased probability
that the track was performed live. A value above 0.8 provides strong
likelihood that the track is live.

-\textbf{loudness} : The overall loudness of a track in decibels (dB).
Loudness values are averaged across the entire track and are useful for
comparing relative loudness of tracks. Loudness is the quality of a
sound that is the primary psychological correlate of physical strength
(amplitude). Values typical range between -60 and 0 db.

-\textbf{mode} : Mode indicates the modality (major or minor) of a
track, the type of scale from which its melodic content is derived.
Major is represented by 1 and minor is 0.

-\textbf{speechiness} : Speechiness detects the presence of spoken words
in a track. The more exclusively speech-like the recording (e.g.~talk
show, audio book, poetry), the closer to 1.0 the attribute value. Values
above 0.66 describe tracks that are probably made entirely of spoken
words. Values between 0.33 and 0.66 describe tracks that may contain
both music and speech, either in sections or layered, including such
cases as rap music. Values below 0.33 most likely represent music and
other non-speech-like tracks.

-\textbf{tempo} : The overall estimated tempo of a track in beats per
minute (BPM). In musical terminology, tempo is the speed or pace of a
given piece and derives directly from the average beat duration.

-\textbf{time\_signature} : An estimated overall time signature of a
track. The time signature (meter) is a notational convention to specify
how many beats are in each bar (or measure).

-\textbf{valence} : A measure from 0.0 to 1.0 describing the musical
positiveness conveyed by a track. Tracks with high valence sound more
positive (e.g.~happy, cheerful, euphoric), while tracks with low valence
sound more negative (e.g.~sad, depressed, angry).

Throughout the course of this worksheet , our response variable is
energy. We shall try and apply the concepts learnt in class to predict
the energy of a song using the other features of a song.

\hypertarget{libraries-used}{%
\subsection{Libraries used}\label{libraries-used}}

-tidyverse

-corrplot

-olsrr :
\href{https://www.rdocumentation.org/packages/olsrr/versions/0.5.3}{documentation}

\hypertarget{points}{%
\subsection{Points}\label{points}}

The problems for this worksheet is for a total of 10 points and the
weightage is not uniformly distributed.

\begin{itemize}
\tightlist
\item
  \emph{Problem 1} : 0.5 points
\item
  \emph{Problem 2} : 2 points
\item
  \emph{Problem 3} : 2 points
\item
  \emph{Problem 4} : 1 point
\item
  \emph{Problem 5} : 1.5 points
\item
  \emph{Problem 6} : 1 point
\item
  \emph{Problem 7} : 2 points
\end{itemize}

\hypertarget{loading-the-dataset}{%
\subsection{Loading the Dataset}\label{loading-the-dataset}}

After downloading the dataset and ensuring the working directory is
right , we read the csv into the dataframe.

\begin{Shaded}
\begin{Highlighting}[]
\FunctionTok{library}\NormalTok{(tidyverse)}
\NormalTok{spotify\_df }\OtherTok{\textless{}{-}} \FunctionTok{read\_csv}\NormalTok{(}\StringTok{\textquotesingle{}spotify.csv\textquotesingle{}}\NormalTok{)}
\end{Highlighting}
\end{Shaded}

\hypertarget{problem-1-0.5-points}{%
\subsection{Problem-1 (0.5 Points)}\label{problem-1-0.5-points}}

Check for missing values in the dataset and normalize the dataset.

\hypertarget{problem-2-2-points}{%
\subsection{Problem-2 (2 Points)}\label{problem-2-2-points}}

Fit a linear model to predict the \emph{energy} rating using \emph{all}
other attributes.Get the summary of the model and explain the results in
detail.{[}\emph{Hint} : Use the lm() function.
\href{https://www.rdocumentation.org/packages/stats/versions/3.6.2/topics/lm}{Click
here} To get the documentation of the same.{]}

\hypertarget{problem-3-2-points}{%
\subsection{Problem-3 (2 points)}\label{problem-3-2-points}}

With the help of a correlogram and scatter plots, choose the features
you think are important and model an MLR. Justify your choice and
explain the new findings.

\hypertarget{problem-4-1-point}{%
\subsection{Problem-4 (1 Point)}\label{problem-4-1-point}}

Conduct a partial F-test to determine if the attributes not chosen by
you in \emph{Problem-3} are significant to predict the energy.What are
the null and alternate hypotheses? {[} \emph{Hint} : Use the anova
function between models created in \emph{Problem-2} and
\emph{Problem-3}{]}

\hypertarget{problem-5-1.5-points}{%
\subsection{Problem-5 (1.5 Points)}\label{problem-5-1.5-points}}

AIC - Akaike Information Criterion is used to compare different models
and determine the best fit for the data. The best-fit model according to
AIC is the one that explains greatest amount of variation using the
fewest number of attributes. Check
\href{https://www.scribbr.com/statistics/akaike-information-criterion/}{this}
resource to learn more about AIC.

Build a model based on AIC using Stepwise AIC regression.Elucidate your
observations from the new model. ( \emph{Hint} : Use an appropriate
function in
\href{https://www.rdocumentation.org/packages/olsrr/versions/0.5.3}{olsrr}
package.)

\hypertarget{problem-6-1-point}{%
\subsection{Problem-6 (1 Point)}\label{problem-6-1-point}}

Plot the residuals of the models built till now and comment on it
satisfying the assumptions of MLR.

\hypertarget{problem-7-2-points}{%
\subsection{Problem-7 (2 Points)}\label{problem-7-2-points}}

For the model built in \textbf{\emph{Problem-2}} , determine the
presence of multicollinearity using VIF. Determine if there are outliers
in the data using
\href{https://www.statisticshowto.com/cooks-distance/}{Cook's Distance}.
If you find any , remove the outliers and fit the model for
\emph{Problem-2} and see if the fit improves. {[} \emph{Hint} : All the
relevant functions can be found in \emph{olsrr} package. An observation
can be termed as an outlier if it has a Cook's distance of more than 4/n
where n is the number of records.{]}

\end{document}
